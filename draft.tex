\documentclass[12pt,twoside,a4paper]{report}
\usepackage{amsmath}
\usepackage{amssymb}
\usepackage{graphicx}
\usepackage{fancyhdr}
\usepackage{enumerate}
\usepackage[vmargin=2.5cm, hmargin=2.5cm]{geometry}
\usepackage{hyperref}
\usepackage{fontspec}
\usepackage{unicode-math}
\usepackage{stackengine}
\usepackage{minitoc}
\usepackage{amsthm}

\setmainfont{Linux Libertine O}
\setsansfont{Linux Biolinum O}


\setlength{\parindent}{0em}
\addtolength{\parskip}{1ex}

\usepackage[backend=biber,style=ieee]{biblatex}
\addbibresource{ref.bib}

\theoremstyle{definition}
\newtheorem*{definition*}{Definition}

\newcounter{motivation}
\renewcommand{\themotivation}{\Roman{motivation}}

\newcommand{\motivation}[1]{%
    \refstepcounter{motivation}%
    \vspace{1.5em}%
    \noindent\textbf{Motivation \themotivation.  #1}
    \par
    \expandafter\edef\csname savedmotivation\themotivation\endcsname{%
        \noexpand\noindent\noexpand\textbf{Motivation \themotivation. #1}\noexpand\par
    }%
}

\begin{document}

\dominitoc

\title{Using functor categories to generate intermediate code with Agda}
\author{Jack Gao}
\date{\today}
\maketitle

\tableofcontents
\newpage

\chapter{Introduction}
    \minitoc
    An Algol-like language is a typed lambda calculus with store. In this dissertation, the source language is an Algol-like language with the following primitive types:
    \begin{itemize}
        \item 
            \textbf{comm}: the commands
        \item 
            \textbf{intexp}: the integer expressions
        \item 
            \textbf{intacc}: the integer acceptors
        \item 
            \textbf{intvar}: the integer variables
    \end{itemize}
    and the set $\Theta$ of types is defined as follows:
    \[ \Theta := \textbf{comm} \mid \textbf{intexp} \mid \textbf{intacc} \mid \textbf{intvar} \mid \Theta \to \Theta \]

    [Considering adding an example of Algol-like language here]

    The target language is an assembly-style intermediate language for a stack machine. It is defined with four stack-descriptor-indexed families of non-terminals: 
    \begin{itemize}
        \item 
            $\langle\text{L}_{sd}\rangle$: lefthand sides
        \item 
            $\langle\text{S}_{sd}\rangle$: Simple righthand sides
        \item
            $\langle\text{R}_{sd}\rangle$: righthand sides
        \item
            $\langle\text{I}_{sd}\rangle$: instruction sequences
    \end{itemize}

    The grammer of the target language is specified in Chapter 3.

    This dissertation presents an implementation of a compiler from the source language to the target language with Agda \cite{Agda}. This implementation is based on the work of Reynolds \cite{Reynolds}, who presented a denotational semantics of Algol-like languages in the form of a presheaf category over stack descriptors. The compiler is implemented as a functor from the source language to the target language. The implementation is verified with Agda's type system, which ensures that the generated code is well-typed and adheres to the semantics of both the source and target languages. This implementation proves and refines Reynolds' work, providing a practical example of how to use functor categories to generate intermediate code.

    \section{Motivation and related work}
        The denotational semantics of Algol-like languages can be structured as a presheaf category over stack descriptors, which has been shown by Reynolds \cite{essence} and Oles \cite{Oles_1} \cite{Oles_2}. By interpreting the source language into this category, where objects of the category represent instruction sequences parameterised by stack layouts, the semantic model directly yields a compiler. The mathematical structure of the compiler has been specified by Reynolds in his paper ``Using Functor Categories to Generate Intermediate Code'' \cite{Reynolds}.

        This project is motivated and guided by the following:

        \motivation{Implementation of the compiler}
        Reynolds concluded that he did not have a proper dependently typed programming language in hand, so his compiler remained a partial function theoretically. We aim to provide a computer implementation of this theoretical framework in a dependently typed programming language.
        

        \motivation{Formal verification of the compiler}
        The terms in Reynolds' work are also written by hand. Terms are complicated and error-prone, and it is difficult to verify the correctness of the terms. We aim to provide a formalisation of the terms in a proof assistant to verify the correctness of the terms.
        
        \motivation{Trend of verified compilers}
        The rise of verified compilers including CompCert \cite{CompCert}, CakeML \cite{CakeML} reflects a broader trend toward trustworthy systems, where correctness proofs replace testing for critical guarantees. Like Lean 4 \cite{Lean4}, we leverage dependent types to internalise the verification of correctness of terms.

    
    \section{Language choice: Agda's advantages}
        Agda \cite{Agda} is a dependently typed programming language and proof assistant. Agda captures the source language's intrinsic sytax with indexed families, which contains only well-typed terms. Therefore, it focuses on the correct programs and rules out the ill-typed nonsensical inputs. 

        (Example of Agda)

        Dependently typed languages provide a natural framework for expressing functor categories is proven both theoretically and practically. There have been dependent-type-theoretic model of categories \cite{Dybjer}, and it has been shown that functor categories arise naturally as dependent function types \cite{Jacobs}. A formalisation of Category Theory, including Cartesian Closed Categories, functors and presheaves has been developed in Agda by Hu and Caratte \cite{Cat_Agda}. Other proof assistants, such as Isabelle/Hol, does not have a dependently typed language structure, and thus cannot express the functor categories as naturally as Agda.

        [Do I need a table for comparing other proof assistants and Agda?]

    \section{Contributions}
        Addressed the two motivations presented in 1.1 and contributed to the following:
        \begin{quote}
            \savedmotivationI
            We implemented the compiler from the source language to the target language in Agda. 

            \savedmotivationII
            We formalised the terms in the source language and target language in Agda, and proved that the compiler is a functor from the source language to the target language. 

            \savedmotivationIII
            This implementation follows the trend of verified compilers, and provides a practical example of how to use functor categories to generate intermediate code.
        \end{quote}


\chapter{Preparation}
    \minitoc

    \section{Starting Point}
    Prior to this project, I had no experience with Agda. Although I was aware of the open-source online tutorial \textit{Programming Language Foundations in Agda} (PLFA) \cite{plfa}, my preparation was limited to setting up the Agda environment on my laptop by following the ``Front Matter'' section of the tutorial.

    I did not have any other experience with compiler beyond Part IB Compiler Construction Course. I had no prior exposure to category theory and type theory before the Part II lectures.

    \section{Category theory background}
        Category 
        \subsection{Category}
        \begin{definition*}[Category]
            A category $\mathcal{C}$ is specified by the following:
            \begin{itemize}
                \item 
                    a collection of objects $\textbf{obj}(\mathcal{C})$, whose elements are called $\mathcal{C}$-objects;
                \item 
                    for each $X, Y \in \textbf{obj}(\mathcal{C})$, a collection of morphisms $\mathcal{C}{(X,Y)}$, whose elements are called $\mathcal{C}$-morphisms from $X$ to $Y$;
                \item 
                    for each $X \in \textbf{obj}(\mathcal{C})$, an element $\textbf{id}_X \in \mathcal{C}{(X,X)}$ called the identity morphism on $X$;
                \item 
                    for each $X, Y, Z \in \textbf{obj}(\mathcal{C})$, a function 
                    \[\begin{aligned}
                        \mathcal{C}{(X,Y)} \times \mathcal{C}{(Y,Z)} &\to \mathcal{C}{(X,Z)} \\
                        (f,g) &\mapsto g \circ f
                    \end{aligned}\]
                    called the composition of morphisms;
            \end{itemize}
            satisfying the following properties:
            \begin{itemize}
                \item 
                    \textbf{(Unit Law)}
                    for all $X, Y \in \textbf{obj}(\mathcal{C})$ and $f \in \mathcal{C}{(X,Y)}$, we have:
                    \begin{equation} \label{law: unit}
                        \textbf{id}_Y \circ f = f = f \circ \textbf{id}_X
                    \end{equation}
                \item
                    \textbf{(Associativity Law)}
                    for all $X, Y, Z, W \in \textbf{obj}(\mathcal{C})$ and $f \in \mathcal{C}{(X,Y)}$, $g \in \mathcal{C}{(Y,Z)}$, $h \in \mathcal{C}{(Z,W)}$, we have:
                    \begin{equation} \label{law: associativity}
                        h \circ (g \circ f) = (h \circ g) \circ f
                    \end{equation}
            \end{itemize}

            [Considering adding different notations for morphisms, composition, identity morphisms, etc.]
            [Considering adding a diagram for the definition of category]
            [Considering adding examples of categories]
            
        \end{definition*}

        \subsection{Functor}
        \begin{definition*}[Functor]
            A functor $F$ from a category $\mathcal{C}$ to a category $\mathcal{D}$ is specified by the following:
            \begin{itemize}
                \item 
                    a function 
                    \[\begin{aligned}
                        \textbf{obj}(\mathcal{C}) &\to \textbf{obj}(\mathcal{D}) \\
                        X &\mapsto F(X)
                    \end{aligned}\]

                \item 
                    for each $X, Y \in \textbf{obj}(\mathcal{C})$, a function 
                    \[\begin{aligned}
                        \mathcal{C}{(X,Y)} &\to \mathcal{D}{(F(X),F(Y))} \\
                        f &\mapsto F(f)
                    \end{aligned}\]
            \end{itemize}
            satisfying the following properties:
            \begin{itemize}
                \item 
                    for all $X, Y \in \textbf{obj}(\mathcal{C})$ and $f \in \mathcal{C}{(X,Y)}$, we have:
                    \begin{equation} \label{law: functor_id}
                        F(\textbf{id}_X) = \textbf{id}_{F(X)}
                    \end{equation}
                \item
                    for all $X, Y, Z \in \textbf{obj}(\mathcal{C})$ and $f \in \mathcal{C}{(X,Y)}$, $g \in \mathcal{C}{(Y,Z)}$, we have:
                    \begin{equation} \label{law: functor_comp}
                        F(g \circ f) = F(g) \circ F(f)
                    \end{equation}
            \end{itemize}
        \end{definition*}
        [Considering adding examples of functors: e.g. free functor, forgetful functor, etc.]

        \subsection{Natural transformation}
        Before introducing the definition of natural transformation, we need to get familiar with the notation of commutative diagrams.


        \subsection{Presheaf}

        \subsection{Functor category}

        \subsection{Cartesian closed category}

        \subsection{Yoneda lemma}

        \subsection{Functor categories}

    \section{Agda}

    \section{Requirement Analysis}

    \section{Tools Used}
    Completing the project is an iterative process. I used Git \cite{git} for version control, and work had been synchronised with a GitHub \cite{github} repository for backup.

    For the development environment, I tried both Emacs \cite{emacs} and Visual Studio Code \cite{vscode} with an agda-mode extension \cite{agda_mode} on Windows Subsystem for Linux with Ubuntu \cite{wsl_ubuntu} 22.04 LTS. I am more familiar with the snippet and syntax highlighting features of Visual Studio Code, so I used it for most of the development. 
    
    The interactive development of Agda with holes and type checking is very useful for debugging. 

    Code from the PLFA tutorial and Agda standard library \cite{agda_std} were used as references. 

    \begin{enumerate}
        \item 
            Tools used:
            git version control

            have tried emacs
            vscode interactive 
    \end{enumerate}



    \printbibliography
\end{document}